% Generated by Sphinx.
\def\sphinxdocclass{report}
\documentclass[letterpaper,10pt,english]{sphinxmanual}
\usepackage[utf8]{inputenc}
\DeclareUnicodeCharacter{00A0}{\nobreakspace}
\usepackage[T1]{fontenc}
\usepackage{babel}
\usepackage{times}
\usepackage[Bjarne]{fncychap}
\usepackage{longtable}
\usepackage{sphinx}
\usepackage{multirow}


\title{pygeo Documentation}
\date{June 18, 2012}
\release{1.2}
\author{Brendan Smithyman}
\newcommand{\sphinxlogo}{}
\renewcommand{\releasename}{Release}
\makeindex

\makeatletter
\def\PYG@reset{\let\PYG@it=\relax \let\PYG@bf=\relax%
    \let\PYG@ul=\relax \let\PYG@tc=\relax%
    \let\PYG@bc=\relax \let\PYG@ff=\relax}
\def\PYG@tok#1{\csname PYG@tok@#1\endcsname}
\def\PYG@toks#1+{\ifx\relax#1\empty\else%
    \PYG@tok{#1}\expandafter\PYG@toks\fi}
\def\PYG@do#1{\PYG@bc{\PYG@tc{\PYG@ul{%
    \PYG@it{\PYG@bf{\PYG@ff{#1}}}}}}}
\def\PYG#1#2{\PYG@reset\PYG@toks#1+\relax+\PYG@do{#2}}

\def\PYG@tok@gd{\def\PYG@tc##1{\textcolor[rgb]{0.63,0.00,0.00}{##1}}}
\def\PYG@tok@gu{\let\PYG@bf=\textbf\def\PYG@tc##1{\textcolor[rgb]{0.50,0.00,0.50}{##1}}}
\def\PYG@tok@gt{\def\PYG@tc##1{\textcolor[rgb]{0.00,0.25,0.82}{##1}}}
\def\PYG@tok@gs{\let\PYG@bf=\textbf}
\def\PYG@tok@gr{\def\PYG@tc##1{\textcolor[rgb]{1.00,0.00,0.00}{##1}}}
\def\PYG@tok@cm{\let\PYG@it=\textit\def\PYG@tc##1{\textcolor[rgb]{0.25,0.50,0.56}{##1}}}
\def\PYG@tok@vg{\def\PYG@tc##1{\textcolor[rgb]{0.73,0.38,0.84}{##1}}}
\def\PYG@tok@m{\def\PYG@tc##1{\textcolor[rgb]{0.13,0.50,0.31}{##1}}}
\def\PYG@tok@mh{\def\PYG@tc##1{\textcolor[rgb]{0.13,0.50,0.31}{##1}}}
\def\PYG@tok@cs{\def\PYG@tc##1{\textcolor[rgb]{0.25,0.50,0.56}{##1}}\def\PYG@bc##1{\colorbox[rgb]{1.00,0.94,0.94}{##1}}}
\def\PYG@tok@ge{\let\PYG@it=\textit}
\def\PYG@tok@vc{\def\PYG@tc##1{\textcolor[rgb]{0.73,0.38,0.84}{##1}}}
\def\PYG@tok@il{\def\PYG@tc##1{\textcolor[rgb]{0.13,0.50,0.31}{##1}}}
\def\PYG@tok@go{\def\PYG@tc##1{\textcolor[rgb]{0.19,0.19,0.19}{##1}}}
\def\PYG@tok@cp{\def\PYG@tc##1{\textcolor[rgb]{0.00,0.44,0.13}{##1}}}
\def\PYG@tok@gi{\def\PYG@tc##1{\textcolor[rgb]{0.00,0.63,0.00}{##1}}}
\def\PYG@tok@gh{\let\PYG@bf=\textbf\def\PYG@tc##1{\textcolor[rgb]{0.00,0.00,0.50}{##1}}}
\def\PYG@tok@ni{\let\PYG@bf=\textbf\def\PYG@tc##1{\textcolor[rgb]{0.84,0.33,0.22}{##1}}}
\def\PYG@tok@nl{\let\PYG@bf=\textbf\def\PYG@tc##1{\textcolor[rgb]{0.00,0.13,0.44}{##1}}}
\def\PYG@tok@nn{\let\PYG@bf=\textbf\def\PYG@tc##1{\textcolor[rgb]{0.05,0.52,0.71}{##1}}}
\def\PYG@tok@no{\def\PYG@tc##1{\textcolor[rgb]{0.38,0.68,0.84}{##1}}}
\def\PYG@tok@na{\def\PYG@tc##1{\textcolor[rgb]{0.25,0.44,0.63}{##1}}}
\def\PYG@tok@nb{\def\PYG@tc##1{\textcolor[rgb]{0.00,0.44,0.13}{##1}}}
\def\PYG@tok@nc{\let\PYG@bf=\textbf\def\PYG@tc##1{\textcolor[rgb]{0.05,0.52,0.71}{##1}}}
\def\PYG@tok@nd{\let\PYG@bf=\textbf\def\PYG@tc##1{\textcolor[rgb]{0.33,0.33,0.33}{##1}}}
\def\PYG@tok@ne{\def\PYG@tc##1{\textcolor[rgb]{0.00,0.44,0.13}{##1}}}
\def\PYG@tok@nf{\def\PYG@tc##1{\textcolor[rgb]{0.02,0.16,0.49}{##1}}}
\def\PYG@tok@si{\let\PYG@it=\textit\def\PYG@tc##1{\textcolor[rgb]{0.44,0.63,0.82}{##1}}}
\def\PYG@tok@s2{\def\PYG@tc##1{\textcolor[rgb]{0.25,0.44,0.63}{##1}}}
\def\PYG@tok@vi{\def\PYG@tc##1{\textcolor[rgb]{0.73,0.38,0.84}{##1}}}
\def\PYG@tok@nt{\let\PYG@bf=\textbf\def\PYG@tc##1{\textcolor[rgb]{0.02,0.16,0.45}{##1}}}
\def\PYG@tok@nv{\def\PYG@tc##1{\textcolor[rgb]{0.73,0.38,0.84}{##1}}}
\def\PYG@tok@s1{\def\PYG@tc##1{\textcolor[rgb]{0.25,0.44,0.63}{##1}}}
\def\PYG@tok@gp{\let\PYG@bf=\textbf\def\PYG@tc##1{\textcolor[rgb]{0.78,0.36,0.04}{##1}}}
\def\PYG@tok@sh{\def\PYG@tc##1{\textcolor[rgb]{0.25,0.44,0.63}{##1}}}
\def\PYG@tok@ow{\let\PYG@bf=\textbf\def\PYG@tc##1{\textcolor[rgb]{0.00,0.44,0.13}{##1}}}
\def\PYG@tok@sx{\def\PYG@tc##1{\textcolor[rgb]{0.78,0.36,0.04}{##1}}}
\def\PYG@tok@bp{\def\PYG@tc##1{\textcolor[rgb]{0.00,0.44,0.13}{##1}}}
\def\PYG@tok@c1{\let\PYG@it=\textit\def\PYG@tc##1{\textcolor[rgb]{0.25,0.50,0.56}{##1}}}
\def\PYG@tok@kc{\let\PYG@bf=\textbf\def\PYG@tc##1{\textcolor[rgb]{0.00,0.44,0.13}{##1}}}
\def\PYG@tok@c{\let\PYG@it=\textit\def\PYG@tc##1{\textcolor[rgb]{0.25,0.50,0.56}{##1}}}
\def\PYG@tok@mf{\def\PYG@tc##1{\textcolor[rgb]{0.13,0.50,0.31}{##1}}}
\def\PYG@tok@err{\def\PYG@bc##1{\fcolorbox[rgb]{1.00,0.00,0.00}{1,1,1}{##1}}}
\def\PYG@tok@kd{\let\PYG@bf=\textbf\def\PYG@tc##1{\textcolor[rgb]{0.00,0.44,0.13}{##1}}}
\def\PYG@tok@ss{\def\PYG@tc##1{\textcolor[rgb]{0.32,0.47,0.09}{##1}}}
\def\PYG@tok@sr{\def\PYG@tc##1{\textcolor[rgb]{0.14,0.33,0.53}{##1}}}
\def\PYG@tok@mo{\def\PYG@tc##1{\textcolor[rgb]{0.13,0.50,0.31}{##1}}}
\def\PYG@tok@mi{\def\PYG@tc##1{\textcolor[rgb]{0.13,0.50,0.31}{##1}}}
\def\PYG@tok@kn{\let\PYG@bf=\textbf\def\PYG@tc##1{\textcolor[rgb]{0.00,0.44,0.13}{##1}}}
\def\PYG@tok@o{\def\PYG@tc##1{\textcolor[rgb]{0.40,0.40,0.40}{##1}}}
\def\PYG@tok@kr{\let\PYG@bf=\textbf\def\PYG@tc##1{\textcolor[rgb]{0.00,0.44,0.13}{##1}}}
\def\PYG@tok@s{\def\PYG@tc##1{\textcolor[rgb]{0.25,0.44,0.63}{##1}}}
\def\PYG@tok@kp{\def\PYG@tc##1{\textcolor[rgb]{0.00,0.44,0.13}{##1}}}
\def\PYG@tok@w{\def\PYG@tc##1{\textcolor[rgb]{0.73,0.73,0.73}{##1}}}
\def\PYG@tok@kt{\def\PYG@tc##1{\textcolor[rgb]{0.56,0.13,0.00}{##1}}}
\def\PYG@tok@sc{\def\PYG@tc##1{\textcolor[rgb]{0.25,0.44,0.63}{##1}}}
\def\PYG@tok@sb{\def\PYG@tc##1{\textcolor[rgb]{0.25,0.44,0.63}{##1}}}
\def\PYG@tok@k{\let\PYG@bf=\textbf\def\PYG@tc##1{\textcolor[rgb]{0.00,0.44,0.13}{##1}}}
\def\PYG@tok@se{\let\PYG@bf=\textbf\def\PYG@tc##1{\textcolor[rgb]{0.25,0.44,0.63}{##1}}}
\def\PYG@tok@sd{\let\PYG@it=\textit\def\PYG@tc##1{\textcolor[rgb]{0.25,0.44,0.63}{##1}}}

\def\PYGZbs{\char`\\}
\def\PYGZus{\char`\_}
\def\PYGZob{\char`\{}
\def\PYGZcb{\char`\}}
\def\PYGZca{\char`\^}
\def\PYGZsh{\char`\#}
\def\PYGZpc{\char`\%}
\def\PYGZdl{\char`\$}
\def\PYGZti{\char`\~}
% for compatibility with earlier versions
\def\PYGZat{@}
\def\PYGZlb{[}
\def\PYGZrb{]}
\makeatother

\begin{document}

\maketitle
\tableofcontents
\phantomsection\label{index::doc}


Contents:


\chapter{pygeo.segyread}
\label{segyread:welcome-to-pygeo-s-documentation}\label{segyread:pygeo-segyread}\label{segyread::doc}
The {\hyperref[segyread:module-pygeo.segyread]{\code{pygeo.segyread}}} submodule is designed to allow interaction with geophysical (seismic) datafiles that use the \href{http://en.wikipedia.org/wiki/SEG-Y}{SEG-Y format}.  The primary purpose of the package is to allow \emph{read-only} access to the SEG-Y data format, though several provisions are made for creating or updating SEG-Y files.
\phantomsection\label{segyread:module-pygeo.segyread}\index{pygeo.segyread (module)}\phantomsection\label{segyread:module-pygeo.segyread}\index{pygeo.segyread (module)}

\section{SEGYFile}
\label{segyread:segyfile}
The  {\hyperref[segyread:pygeo.segyread.SEGYFile]{\code{SEGYFile}}} class represents the SEG-Y or SU datafile efficiently, and initially loads only the metadata necessary to set certain parameters, viz: filesize, endian, data format.  Several objects are created inside the namespace of the {\hyperref[segyread:pygeo.segyread.SEGYFile]{\code{SEGYFile}}} object, viz: \textbf{thead}, \textbf{bhead}, \textbf{trhead}, \textbf{endian}, \textbf{mendian}, \textbf{ns}, \textbf{ntr}, \textbf{filesize}, \textbf{ensembles}.
\index{SEGYFile (class in pygeo.segyread)}

\begin{fulllineitems}
\phantomsection\label{segyread:pygeo.segyread.SEGYFile}\pysigline{\strong{class }\code{pygeo.segyread.}\bfcode{SEGYFile}}
Provides read access to a SEG-Y dataset (headers and data).
\begin{quote}\begin{description}
\item[{Parameters}] \leavevmode\begin{itemize}
\item {} 
\textbf{filename} (\href{http://docs.python.org/library/functions.html\#str}{\emph{str}}) -- The system path of the SEG-Y file to open.

\item {} 
\textbf{verbose} (\href{http://docs.python.org/library/functions.html\#bool}{\emph{bool}}) -- Controls whether diagnostic information is printed.  This includes status messages when endian and format conversions are made, and may be useful in diagnosing problems.

\item {} 
\textbf{majorheadersonly} (\href{http://docs.python.org/library/functions.html\#bool}{\emph{bool}}) -- Only read certain specific headers (legacy).  No longer relevant, but may be expected by some old programs.

\item {} 
\textbf{isSU} (\href{http://docs.python.org/library/functions.html\#bool}{\emph{bool}}) -- Controls whether SEGYFile treats the datafile as a Seismic Unix variant SEG-Y file.  This overrides assumptions for endianness and format, and presumes the absence of the 3200-byte text header and 400-byte binary header.

\item {} 
\textbf{endian} (\href{http://docs.python.org/library/functions.html\#str}{\emph{str}}) -- Allows specification of file endianness {[}Foreign,Native,Little,Big{]}.  By default this is auto-detected using a heuristic method, but it will fail for e.g., SEG-Y files that contain all zeros, or very noisy data.

\item {} 
\textbf{usemmap} (\href{http://docs.python.org/library/functions.html\#bool}{\emph{bool}}) -- Controls whether memory-mapped I/O is used. Default True.  In most (all?) cases this should be more efficient, and will be disabled automatically if not supported.

\end{itemize}

\item[{Returns}] \leavevmode
SEGYFile instance

\item[{Variables}] \leavevmode\begin{itemize}
\item {} 
\textbf{thead} -- \emph{str} -- contains an ASCII-encoded translation of the EBCDIC 3200-byte tape header.

\item {} 
\textbf{bhead} -- \emph{dict} -- contains key:value pairs describing the data in the 400-byte binary reel header.

\item {} 
\textbf{trhead} -- {\hyperref[segyread:pygeo.segyread.SEGYTraceHeader]{\code{SEGYTraceHeader}}} instance -- acts like a list of all the trace headers.  Individual items each return a dictionary that contains key:value pairs describing the data in the trace header.

\item {} 
\textbf{endian} -- \emph{str} -- describing the endian of the datafile.

\item {} 
\textbf{mendian} -- \emph{str} -- autodetected machine endian.

\item {} 
\textbf{ns} -- \emph{int} -- number of samples in each trace.

\item {} 
\textbf{ntr} -- \emph{int} -- number of traces in dataset.

\item {} 
\textbf{filesize} -- \emph{int} -- size of datafile in bytes.

\item {} 
\textbf{ensembles} -- \emph{dict} -- only exists if the experimental function \code{SEGYFile.\_calcEnsembles()} is called.  Maps shot gather numbers to trace numbers.  \emph{Experimental}

\end{itemize}

\end{description}\end{quote}
\index{\_\_getitem\_\_() (pygeo.segyread.SEGYFile method)}

\begin{fulllineitems}
\phantomsection\label{segyread:pygeo.segyread.SEGYFile.__getitem__}\pysiglinewithargsret{\bfcode{\_\_getitem\_\_}}{}{}
Returns traces from the open seismic dataset, with support for standard
Python slice notation.  Trace numbers are zero-based.
\begin{quote}\begin{description}
\item[{Parameters}] \leavevmode
\textbf{index} -- Slice object or trace number (using zero-based numbering).

\item[{Returns}] \leavevmode
ndarray -- 2D array containing (possibly non-adjacent) seismic traces

\end{description}\end{quote}

\end{fulllineitems}

\index{findTraces() (pygeo.segyread.SEGYFile method)}

\begin{fulllineitems}
\phantomsection\label{segyread:pygeo.segyread.SEGYFile.findTraces}\pysiglinewithargsret{\bfcode{findTraces}}{}{}
Finds traces whose header values fall within a particular range.  Trace numbers are 1-based, i.e., for use with readTraces.
\begin{quote}\begin{description}
\item[{Parameters}] \leavevmode\begin{itemize}
\item {} 
\textbf{key} (\href{http://docs.python.org/library/functions.html\#str}{\emph{str}}) -- Key value of trace header to scan (uses lower-case SU names; see TRHEADLIST.

\item {} 
\textbf{kmin} (\href{http://docs.python.org/library/functions.html\#int}{\emph{int}}) -- Minimum key value (inclusive).

\item {} 
\textbf{kmax} (\href{http://docs.python.org/library/functions.html\#int}{\emph{int}}) -- Maximum key value (inclusive).

\end{itemize}

\end{description}\end{quote}

\end{fulllineitems}

\index{readTraces() (pygeo.segyread.SEGYFile method)}

\begin{fulllineitems}
\phantomsection\label{segyread:pygeo.segyread.SEGYFile.readTraces}\pysiglinewithargsret{\bfcode{readTraces}}{}{}
Returns trace data as a list of numpy arrays (i.e. non-adjacent trace
numbers are allowed). Requires that traces be fixed length.
\begin{quote}\begin{description}
\item[{Parameters}] \leavevmode
\textbf{traces} (\emph{list, None}) -- List of traces to return, using 1-based trace numbering.  Optional; if omitted, all traces are returned.

\item[{Returns}] \leavevmode
ndarray -- 2D array containing (possibly non-adjacent) seismic traces

\end{description}\end{quote}
Changed in version devel.
This is now a legacy interface, and is superseded by the \_\_getitem\_\_
interface, which uses standard Python slice notation.

\end{fulllineitems}

\index{sNormalize() (pygeo.segyread.SEGYFile method)}

\begin{fulllineitems}
\phantomsection\label{segyread:pygeo.segyread.SEGYFile.sNormalize}\pysiglinewithargsret{\bfcode{sNormalize}}{}{}
Utility function that takes seismic traces and returns an amplitude
normalized version.
\begin{quote}\begin{description}
\item[{Parameters}] \leavevmode
\textbf{traces} (\emph{ndarray, list}) -- List or array of traces to normalize.

\end{description}\end{quote}

\end{fulllineitems}

\index{writeFlat() (pygeo.segyread.SEGYFile method)}

\begin{fulllineitems}
\phantomsection\label{segyread:pygeo.segyread.SEGYFile.writeFlat}\pysiglinewithargsret{\bfcode{writeFlat}}{}{}
Outputs seismic traces as a flat file in IEEE floating point and
native endian.
\begin{quote}\begin{description}
\item[{Parameters}] \leavevmode
\textbf{outfilename} (\href{http://docs.python.org/library/functions.html\#str}{\emph{str}}) -- Filename for new flat datafile.

\end{description}\end{quote}

\emph{Experimental}

\end{fulllineitems}

\index{writeSEGY() (pygeo.segyread.SEGYFile method)}

\begin{fulllineitems}
\phantomsection\label{segyread:pygeo.segyread.SEGYFile.writeSEGY}\pysiglinewithargsret{\bfcode{writeSEGY}}{}{}
Outputs seismic traces in a new SEG-Y file, optionally using the headers
from the existing dataset.
\begin{quote}\begin{description}
\item[{Parameters}] \leavevmode\begin{itemize}
\item {} 
\textbf{outfilename} (\href{http://docs.python.org/library/functions.html\#str}{\emph{str}}) -- Filename for new SEG-Y datafile.

\item {} 
\textbf{traces} (\emph{ndarray, list}) -- Array of seismic traces to output.

\item {} 
\textbf{headers} (\emph{list, None}) -- List of three headers: {[}thead, bhead, trhead{]}.  If omitted, the existing headers in the SEGYFile instance are used. \emph{thead} is an ASCII-formatted 3200-byte text header. \emph{bhead} is a list of binary header values similar to SEGYFile.bhead.  \emph{trhead} is a list or list-like object of trace header values.

\end{itemize}

\end{description}\end{quote}

\end{fulllineitems}

\index{writeSU() (pygeo.segyread.SEGYFile method)}

\begin{fulllineitems}
\phantomsection\label{segyread:pygeo.segyread.SEGYFile.writeSU}\pysiglinewithargsret{\bfcode{writeSU}}{}{}
Outputs seismic traces in a new CWP SU file, optionally using the headers
from the existing dataset.
\begin{quote}\begin{description}
\item[{Parameters}] \leavevmode\begin{itemize}
\item {} 
\textbf{outfilename} (\href{http://docs.python.org/library/functions.html\#str}{\emph{str}}) -- Filename for new SU datafile.

\item {} 
\textbf{traces} (\emph{ndarray, list}) -- Array of seismic traces to output.

\item {} 
\textbf{trhead} (\emph{list, None}) -- List or list-like object of trace header values.  If omitted, the existing headers in the SEGYFile instance are used.

\end{itemize}

\end{description}\end{quote}

\end{fulllineitems}


\end{fulllineitems}



\section{SEGYTraceHeader}
\label{segyread:segytraceheader}
The {\hyperref[segyread:pygeo.segyread.SEGYTraceHeader]{\code{SEGYTraceHeader}}} class efficiently indexes the trace headers of the parent {\hyperref[segyread:pygeo.segyread.SEGYFile]{\code{SEGYFile}}}.  This makes it possible to access the headers of an individual trace, or a series of traces without prefetching them from the file on disk.  It interfaces directly with the conventional or memory-mapped file object inside the {\hyperref[segyread:pygeo.segyread.SEGYFile]{\code{SEGYFile}}} object.
\index{SEGYTraceHeader (class in pygeo.segyread)}

\begin{fulllineitems}
\phantomsection\label{segyread:pygeo.segyread.SEGYTraceHeader}\pysigline{\strong{class }\code{pygeo.segyread.}\bfcode{SEGYTraceHeader}}
Provides read access to trace headers from an existing {\hyperref[segyread:pygeo.segyread.SEGYFile]{\code{SEGYFile}}} instance.
\begin{quote}\begin{description}
\item[{Parameters}] \leavevmode\begin{itemize}
\item {} 
\textbf{sf} -- Parent class to attach to.

\item {} 
\textbf{sf} -- {\hyperref[segyread:pygeo.segyread.SEGYFile]{\code{SEGYFile}}}

\end{itemize}

\item[{Returns}] \leavevmode
{\hyperref[segyread:pygeo.segyread.SEGYTraceHeader]{\code{SEGYTraceHeader}}} instance

\end{description}\end{quote}
\index{\_\_getitem\_\_() (pygeo.segyread.SEGYTraceHeader method)}

\begin{fulllineitems}
\phantomsection\label{segyread:pygeo.segyread.SEGYTraceHeader.__getitem__}\pysiglinewithargsret{\bfcode{\_\_getitem\_\_}}{}{}
Returns dictionary (or list of dictionaries) that maps header information
for each defined SEG-Y trace header.  SU style names, see TRHEADLIST.
\begin{quote}\begin{description}
\item[{Parameters}] \leavevmode
\textbf{index} -- Slice object or trace number (using zero-based numbering).

\item[{Returns}] \leavevmode
dict, list

\end{description}\end{quote}

\end{fulllineitems}


\end{fulllineitems}



\chapter{Indices and tables}
\label{index:indices-and-tables}\begin{itemize}
\item {} 
\emph{genindex}

\item {} 
\emph{modindex}

\item {} 
\emph{search}

\end{itemize}


\renewcommand{\indexname}{Python Module Index}
\begin{theindex}
\def\bigletter#1{{\Large\sffamily#1}\nopagebreak\vspace{1mm}}
\bigletter{p}
\item {\texttt{pygeo.segyread}} \emph{(Unix)}, \pageref{segyread:module-pygeo.segyread}
\end{theindex}

\renewcommand{\indexname}{Index}
\printindex
\end{document}
